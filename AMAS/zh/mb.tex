% \documentstyle[twoside,def-head,amstola,epsfig]{ctexart}
\documentclass[twoside,a4paper]{ctexart}
\usepackage{epsfig}
%--------------------------Page Format--------------------------
%\headsep 0.5 true cm
\headsep0.12true cm
\voffset=0pt
\topmargin 0pt
\input amssym.def
\oddsidemargin 0pt
\footskip 2mm
\evensidemargin 0pt
\textheight 21 true cm
\textwidth 14 true cm
\renewcommand\baselinestretch{1.3}
\setcounter{page}{1}
%\parskip 0.2cm
\parindent 2\ccwd
\nofiles
%---------------------------------------------------------------
\begin{document}
%\raggedleft
\abovedisplayskip=10.0pt plus 2.0pt minus 2.0pt
\belowdisplayskip=10.0pt plus 2.0pt minus 2.0pt
%====================================================================
\catcode`@=11
\long\def\@makefntext#1{\parindent 1em\noindent \hbox to 0pt{\hss$^{}$}#1}
%\def\evenfoot{}
%\def\oodfoot{}
\catcode`\@=12
%====================================================================
\input cyracc.def
  \font\tencyr=wncyr10
  \def\cyr{\tencyr\cyracc} %这三条语句用来调用斯拉夫字库, 包括俄文
\font\one=cmbx10 scaled\magstep4
\font\fif=cmti10 scaled\magstep1
\font\fiv=cmti10 scaled\magstep5
\font\ooo=cmsy10 scaled\magstep3
\font\two=cmcsc10
\font\three=cmti8
\font\ss=cmsl8
\font\bb=cmbx8
\font\four=cmbx10 scaled\magstep1
\font\six=cmti10 scaled\magstep1
\font\small=cmr8

\def\ol{\overline}
\def\T{\text{T}}
\def\t{\text}
\def\d{{\text{d}}}
\def\om{\omega}
\def\Om{\Omega}
\def\sub{\subset}
\def\al{\alpha}
\def\dt{\delta}
\def\ep{\varepsilon}
\def\eq{\equiv}
\def\la{\lambda}
\def\lg{\langle}
\def\rg{\rangle}
\def\e{{{\text{e}}}}
\def\R{{\Bbb R}}
\def\C{{\Bbb C}}
\def\A{{\cal A}}
\def\B{{\cal B}}
\def\vp{\varphi}
\def\pt{\partial}
\def\na{\nabla}
\def\Dt{\Delta}
\def\pma{\pmatrix}
\def\epm{\endpmatrix}
\def\Ga{\Gamma}
\def\for{\forall}
\def\si{\sigma}
\def\wt{\widetilde}
\def\tha{\theta}
\def\V{{\cal V}}
\def\be{\beta}
\def\bm{\bmatrix}
\def\ebm{\endbmatrix}
\def\ga{\gamma}
\def\us{\underset}
\def\os{\overset}
\def\1{{\pmb 1}}
\def\tri{\triangle}
\def\Si{\Sigma}
\def\U{{\cal U}}

\def\TT{{\songti\zihao{-5}{刘焕彬, 何穗, 孙六全: 两类新的变参数下降算法及收敛性}}}
%---------------------------------------------------------------
\def\evenhead{{\protect{\zihao{-5}\songti \hfill  应 \quad  用 \quad  数
 \quad  学 \quad  学 \quad  报} \hfill 34\,{\zihao{-5}\songti  卷}}}
\def\oddhead{{\protect 5\,{\zihao{-5}\songti  期} \hfill \TT \hfill}}
%---------------------------------------------------------------

%-----------------------use-------------------

%-----------------------enduse-------------------------
\vspace*{-11.5mm}

\thispagestyle{empty}

\noindent
\hbox to \textwidth{\rm{\zihao{-5}\songti  第}\, 34\, {\zihao{-5}\songti  卷 \ 
 第}\, 5\, {\zihao{-5}\songti  期}\hfill
{\zihao{-4}  应 \  用 \  数 \  学 \  学 \  报}\hfill Vol.\, 34 \ No.\, 5}
\vskip -0.2mm
\par\noindent
\hbox to \textwidth{\rm 2011 \,{\zihao{-5}\songti 年}\, 9\, {\zihao{-5}\songti
 月}\hfill ACTA MATHEMATICAE APPLICATAE SINICA\hfill Sep., 2011}
\vskip -0.3mm
\par\noindent
\rule[1.5mm]{\textwidth}{0.3pt}\hspace*{-\textwidth}\rule[0.7mm] 
{\textwidth}{0.3pt}

%%%%%%%%%%%%%%%%%%%%%%%%%%%%%%%%%%%%%%%%%%%%%%%%%%%%%%%%%%%%%%%%%
\ziju{0.025}

\vbox{\vskip38pt}
\centerline{\heiti\zihao{-2}{复发事件下一般半参数比率回归模型$^{^{^{^{\hbox{$\ast$}}}}}$}}
%\vskip2pt
%\c{\heiti\zihao{-2}{{\one Strum-Liouville}奇异边值问题中的应用$^{^{^{^{\hbox{$\ast$}}}}}$}}
\footnotetext{\songti\zihao{-6}{本文} \small 2006 \songti\zihao{-6}{年} 
1 \songti\zihao{-6}{月} 9 \songti\zihao{-6}{日收到.}  
\small 2007 \songti\zihao{-6}{年} 9 \songti\zihao{-6}{月} 24 
\songti\zihao{-6}{日收到修改稿.}}
\footnotetext{$^{\ast}$ 
\songti\zihao{-6}{国家自然科学基金(10571169,10731010), 国家重点基础研究发展计划(2007CB814902)
以及湖北省高等学校优秀中青年科技创新团队项目经费(03BA85)资助项目.}}
\vskip12pt
\centerline{\fangsong\zihao{-4}{刘焕彬}}
\vskip5pt
\centerline{({\zihao{-6}{黄冈师范学院数学系, 黄冈}} {\small 438000})}
\vskip7pt
\centerline{\fangsong\zihao{-4}{何 \ \ 穗}}
\vskip5pt
\centerline{({\zihao{-6}{华中师范大学数学与统计学学院, 武汉}} {\small 430079})}
\vskip7pt
\centerline{\fangsong\zihao{-4}{孙六全}}
\vskip5pt
\centerline{({\zihao{-6}{中国科学院数学与系统科学研究院应用数学研究所, 北京}} {\small 100190})} 
\vskip2pt
\centerline{({\small E-mail: slq@amt.ac.cn})}

\vskip25pt

{\narrower\songti\zihao{-5}\small 
\noindent{\heiti 摘 \ \  要}\quad {本文在复发事件数据下, 研究了一个一般半参数比率回归模型中参数的估计问题, 
给出了该模型中未知参数和非参数函数的一种估计方法, 并证明了这些估计的相合性和渐近正态性.} 

\vskip5pt

\noindent{\heiti 关键词}\quad 半参数比率模型; 复发事件; 边际回归; 估计方程

\noindent{\heiti\bf MR(2000)主题分类}\ \ 62G05; 62N01 

\noindent{\heiti 中图分类}\ \ O212.7

}

\vskip25pt

\noindent{\four 1 } \ {\heiti\zihao{-4} 引言}

\

\songti\zihao{8}\ziju{0.07}

“文章正文”

\

\centerline{\heiti  参 \quad  考 \quad  文 \quad  献}

\

\begin{thebibliography}{99}
\songti\zihao{-6}\small

\bibitem[1]{ref1} Wang Z Q, Willem M. \ Singular Minimization Problems. \ {\three J. Differential Equations}, 2000, 161(2): 307--320 (杂志的模式)

\bibitem[2]{ref2} Willem M. \ Minimax Theorems. \ Boston: Birkhauser, 1996 (专著的模式)

\bibitem[3]{ref3}   

\end{thebibliography}


\vbox{\vskip15pt}

\centerline{\four A General Class of Semiparametric Rates Models}  
\vskip2pt
\centerline{\four for Recurrent Event Data}
%\vskip2pt
%\c{\four }
\vskip12pt
\centerline{\two LIU Huanbin}
\vskip5pt
\centerline{({\three Department of Mathematics, Huanggang Normal University, Huanggang} {\small 438000})}
\vskip7pt
\centerline{\two HE Sui}
\vskip5pt
\centerline{({\three College of Mathematics and Statistics, Central China Normal University, Wuhan} {\small 430079})}    
\vskip7pt
\centerline{\two SUN Liuquan}
\vskip5pt
\centerline{({\three Academy of Mathematics and System Sciences, Chinese Academy of Sciences,Beijing} {\small 100190})}
\vskip2pt
\centerline{({\three E-mail: slq@amt.ac.cn})}

\

\noindent{\bf Abstract}\quad In this article, we propose a general class of semiparametric rates
 models for recurrent event data, which includes the proportional rates model and
 a semiparametric additive rates model  as special cases. For the inference on the model parameters,
 estimating equation approaches are developed. The consistency and asymptotic normality properties of
 the proposed estimators are established. 

\vskip5pt

\noindent{\bf Key words}\quad semiparametric rates models; recurrent event; marginal regression; 

\hskip1.54cm estimating equation

\noindent{\bf MR(2000) Subject Classification}\quad 62G05; 62N015

\noindent{\bf Chinese Library Classification}\quad O212.7

    

 
 \end{document}


$A=\mathop \cup \limits_k A_k $

\renewcommand{\arraystretch}{0.9}
\renewcommand{\arraycolsep}{1pt}
